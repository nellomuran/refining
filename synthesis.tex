In this section we show how our framework captures generalisations of
the classical \LTL synthesis problem to the context of nondeterministic strategies.

\subsection{Reactive synthesis}
\label{sec-ltl-synth}

We first recall the standard \LTL synthesis problem as defined
in~\cite{pnueli1989synthesisshort}: consider a set of \emph{input
  variables} $I$ controlled by the environment and a set of
\emph{output variables} $O$ controlled by the system. In each round,
first the environment chooses a valuation of the inputs $i_k\in 2^I$
(called \emph{input}), and then the system reacts by choosing a
valuation on the output variables $o_k\in 2^O$ (called \emph{output});
an infinite word over $2^{I\cup O}$ is called an
\emph{execution}. The system has perfect recall, meaning that its
choices can depend on all previous choices of the environment, and a
strategy for the system is thus a function $S:(2^I)^+\to 2^O$. Given
an infinite sequence of inputs $w=i_0i_1i_2\ldots \in (2^I)^\omega$, we
define the execution $S(w)=i_0\cup o_0, i_1\cup o_1, i_2\cup o_2\ldots$ where,
for each $k\geq 0$, $o_k=S(i_0\ldots i_k)$.

The \emph{\LTL synthesis problem} consists in,
given $I$, $O$ and an \LTL formula $\psi$ over atoms $I\cup O$,
synthesising a (finite representation of a) system $S:(2^I)^+\to 2^O$ such
that for all $w=i_0i_1i_2\ldots\in (2^I)^\omega$ it holds that $S(w)\models\psi$.
This problem can easily
be coded in Strategy Logic: one builds a turn-based game arena $\CGS_{I,O}$ (which
can be represented as a \CGS, see Remark~\ref{rem-turn-based}) with
two players, E (for Environment) and S (for System) in
which first the environment chooses an input $i$, then the system chooses
an output $o$, reaching a position labelled with atoms $i\cup o$ and in which it is
the environment's turn to play. The LTL synthesis problem for $(I,O,\psi)$ can
then be solved by model-checking on $\CGS_{I,O}$ the \SLref formula
\[\phisynthdet(\psi):=\existsd \var.   \foralld \varb.
 (S,\var)(E,\varb) \A \psi\]
 Note that this really solves the synthesis
problem as existing model-checking algorithms for Strategy Logic can
synthesise  witness strategies (when they exist) for strategy variables existentially
quantified at the beginning of the formula.
Rewriting $\phisynthdet$ as $\existsd \var. \neg  \existsd  \varb.
 (S,\var)(E,\varb) \E \neg \psi$
we see that it has simulation depth 1 (see
Remark~\ref{rem-def-existsdet}) and thus can be solved by the
model-checking algorithm for \SLref in doubly
exponential time (Theorem~\ref{theo-SLref}), which is optimal since
\LTL synthesis is 2\EXPTIME-complete~\cite{pnueli1989synthesisshort}.

Note also that in the case of deterministic strategies, fixing a
strategy for each player fixes a unique outcome, and thus $\A\psi$ in
the formula above could be replaced by $\E\psi$ without affecting the semantics.
Also, once a deterministic strategy $\var$ is fixed for $S$, each
deterministic strategy $\varb$ for $E$ fixes an outcome of strategy $\var$, and each outcome
of $\var$ can be obtained by fixing a deterministic strategy $\varb$
for $E$. It then follows by the
semantics of the outcome quantifier $\A$ that, when
considering only deterministic strategies,
$\phisynthdet(\psi)$ is equivalent to $\existsd \var (S,\var)\A\psi$.

\subsection{Supervisory control}
\label{sec-nd-synth}

Considering nondeterministic strategies, as we do, does not change
anything for classical \LTL synthesis. Indeed, consider the following
formula:
\[\phisynthndet(\psi):=\exists \var.   \forall \varb.
 (S,\var)(E,\varb) \A \psi\]
which differs from $\phisynthdet(\psi)$ only in that it allows the system
$S$ to use nondeterministic strategies. It is easy to check that:
\begin{proposition}
  \label{prop-equiv-synth}
  For every \LTL formula $\psi$, \[\CGS_{I,O}\models
  \phisynthdet(\psi) \quad\mbox{ iff } \quad\CGS_{I,O}\models \phisynthndet(\psi)\]
\end{proposition}

\begin{figure}
  \centering
\begin{tikzpicture}[shorten >= 1pt, shorten <= 1pt, minimum size  =
  .7cm, auto]
  \tikzstyle{rond}=[circle,draw=black]
  
  \node[rond] (init) {};
  \node (aux-init) [above=.5cm of init] {};
  \node (aux) [below =1.5cm of init] {};
  \node[rond] (1) [left = 1cm of aux] {$p$};
  \node[rond] (2) [right = 1cm of aux] {};
  
  \path[->,thick] 
  (aux-init) edge  (init)       
  (init) edge   node[minimum size=5pt,swap,pos=.6]   {$a,\cdot$}    (1)
  (init) edge  node[minimum size=5pt, pos=.6] {$b,\cdot$} (2);

  \path[->,thick]
    (1) edge [loop left]     node[minimum size=5pt]
    {$\cdot\,,\cdot$} (1)
    (2) edge [loop right]     node[minimum size=5pt]
    {$\cdot\,,\cdot$} (2);
\end{tikzpicture}  
  \caption{A model in which $\phi'_1$ holds}
  \label{fig:example}
\end{figure}

However it makes a difference if, instead of considering only
universal satisfaction of an \LTL formula on all outcomes, we consider
other forms of branching-time specifications, in particular
specifications that require existence of outcomes satisfying different
properties, sometimes called \emph{possibility requirements}~\cite{kupferman1997module}. For instance,
\[\phi_1:=\existsd \var (S,\var) \foralld \varb (E,\varb) (\E \always p
  \wedge \E \F \neg p)\]
is always false, because a pair of
strategies for the system and the environment determine a unique
outcome that cannot satisfy both $\always p$ and $\F\neg p$.
However
\[\phi'_1:=\exists \var (S,\var) \forall \varb (E,\varb) (\E \always p
  \wedge \E \F \neg p)\]
 can be true: for instance in model $\CGS_1$ depicted in Figure~\ref{fig:example}, where the
initial position is marked by an incoming arrow and  '$\cdot$' stands
for any possible action, it is clear that the nondeterministic
strategy that allows both $a$ and $b$ in the initial position is a
witness for the satisfaction of $\phi'_1$.
Note that in this example the environment's choices are irrelevant.

Then we can combine the two kinds of specifications by  requiring the
existence of behaviours satisfying some properties, while requiring
that all  behaviours satisfy some other property.
For instance, formula
\[\phi_2:=\exists \var (S,\var) \forall\varb (E,\varb)(\E \F p \wedge \E \F
  q \wedge \A \always \F (p\vee q))\]
  asks that some behaviours 
 reach $p$, some  reach $q$, and all behaviours go infinitely
 often through $p$ or $q$.

For arbitrary \CTLs specifications $\phi$, the formula $\exists \var
(S,\var) \forall\varb (E,\varb)\phi$ captures the
\emph{supervisory control problem} for \CTLs as defined in
~\cite{KMTV00}, where it is proved to be 3\EXPTIME-complete.
This formula has simulation depth at most 2,
  and thus the model-checking algorithm for \SLref provides the
  optimal upper bound. The same complexity is proved
  in~\cite{jiang2006supervisory} in a slightly different setting.

  % \begin{remark}
  %   \label{rem-det-env}
  %   In~\cite{KMTV00} the supervisory control
  %   problem is stated with \emph{nondeterministic} strategies for the
  %   environment, while in our formulation the strategy of the system
  %   has to satisfy the specification against every
  %   \emph{deterministic} strategy of the environment. The former is
  %   more demanding, as nondeterministic strategies include
  %   deterministic ones; but if we know that the environment has to
  %   follow a deterministic strategy, then the latter variant gives
  %   even more confidence regarding the quality of the system's
  %   strategy, in the sense that it ensures not only that all the
  %   possibilities (formulas of the form $\E\psi$) in the specification
  %   are satisfied, but in addition it means that the system can choose
  %   which one to satisfy. Indeed, if the system's strategy $\sigma$ ensures that against
  %   any  deterministic
  %   strategy of the environment, $\E$ 
  % \end{remark}
  
It is then natural to look for  strategies that allow for as many
behaviours as possible. Such strategies are usually called
\emph{maximally permissive} in the litterature.



\subsection{Maximally permissive synthesis}
\label{sec-max-perm}

Different definitions of maximally permissive strategies have been used
in the literature. In supervisory control
theory~\cite{ramadge1987supervisory}, maximality is expressed in terms
of inclusion of sets of behaviours/outcomes, or equivalently by
referring to simulation between the unfoldings of the systems where
unauthorised transitions have been pruned, as
in~\cite{pinchinat2005you}.  In~\cite{bernet2002permissive},
strategies are also compared by looking at inclusion of the behaviours/outcomes they allow. However, as it is proved
in~\cite{bernet2002permissive}, the existence of maximally permissive
strategies for this notion of maximality is ensured only for simple
safety games.

For this reason an alternative notion of strategy
permissiveness was introduced in~\cite{bouyer2009measuring} for
reachability games and further studied in~\cite{bouyer2011measuring}
for parity games. In this setting, to each transition in a game is
attached a cost that represents the penalty incurred by a strategy
that does not allow this transition, and a maximally permissive
strategy is one that minimises penalties.

The latter definition ensures that
maximally permissive strategies always exist, but the quantitative
aspects involved, which are close to mean-payoff
games~\cite{ehrenfeucht1979positional,gurvich1988cyclic}, are known to
quickly lead to undecidability when introduced in Strategy Logic~\cite{gardy2017semantics}.
As for the definition based on inclusion of outcomes, it makes sense in the
two-player antagonistic setting; but it is not adapted to our multi-player setting, where the set of
outcomes induced by a strategy depends on which agent uses it, and
which other agents have a defined strategy.

For this reason we consider the following natural definition of permissiveness based on refinement
of strategies:
\begin{definition}
  \label{def-permissive}
  Strategy $\sigma'$ is \emph{more permissive than} strategy $\sigma$
  if $\sigma\refinesstr\sigma'$. 
\end{definition}

With this definition, given a formula $\phi(\var)$ we can express that a strategy $\var$
  is maximally permissive with regards to  $\phi(\var)$, \ie, that
  it satisfies $\phi(\var)$ and that no more permissive strategy
  satisfies it. Define formula $\maxperm(\var,\phi)$ as
  follows:
  \[\maxperm(\var,\phi) \quad := \quad \phi(\var) \wedge (\forall \varb \;
    \var \refinesstr \varb \impl \neg \phi(\varb))\]
  For instance, coming back to the framework of reactive synthesis, if
the specification is a  \CTLs formula $\phi$, it holds that
  $\CGS,\assign,\pos_\init\models \maxperm(\var,\forall \varc
  \bind[S]{\var}\bind[E]{\varc}\phi)$ if, and only
  if, $\assign(\var)$ is a maximally permissive system for
  specification $\phi$.

To solve the problem of maximally permissive synthesis for a specification $\phi\in\CTLs$,  we can thus model-check the
following \SLref formula:

\[\phisynthmp(\phi):=\exists\var (\ag,\var) \maxperm(\var,\forall \varc
  \bind[S]{\var}\bind[E]{\varc}\phi)\]

When the formula is true, our model-checking algorithm can also
produce a witness strategy for $\var$ that is maximally permissive.
If $\phi$ is a \CTLs formula, $\phisynthmp$ has simulation depth 3,
and thus we obtain a 4\EXPTIME upper-bound for this problem, which can
be reduced to 3\EXPTIME for \LTL specifications, \ie, for formulas of
the form $\phi=\A\psi$ with $\psi\in\LTL$.



\subsection{Strategy refinement}
\label{sec-plan-B}

One problem that is of interest in AI is
that of producing plans that enforce some safety property, and in addition
can at any time be refined to reach some secondary goal \cite{WrightMN18,PercassiG19}. For instance,
consider an electric vehicle transporting rocks from a point A, where
they are cut, to a point B, where they are used. The truck must ensure
that the stock of rocks at point B never runs out (safety property). We would like to synthesise a
strategy for the truck such that  this property is satisfied, but also
so that at any time, the strategy can be refined to make the truck go
through point C, where its battery can be reloaded.
We can express  this problem in \SLref as follows, where ``empty''
holds when there are no more rocks in point B, and ``reload'' means
that the truck is at point C, reloading its battery.

\[\exists\var (\text{truck},\var) \A \always \left ( \neg \text{empty} \wedge
  (\exists \varb. \varb\refines\var \wedge (\text{truck},\varb)\A\F \text{reload}) \right )\]


\subsection{Module checking}
\label{section:module}
Module checking~\cite{kupferman2001module} is a generalisation of model checking to the setting
of open systems, \ie, systems that interact with an environment. The
idea is to check that the system satisfies a given property, specified
for instance in \LTL or \CTLs, for any possible environment in which
it is used.

In this setting, the environment is seen as an entity that can cut
some transitions of the system. One particular environment can thus be
modelled as a nondeterministic strategy that defines which transitions
are allowed, and the module checking problem for a property $\phi$ can
be written as follows in \SLref, where $E$ is the environment:

\[\forall\var (E,\var) \,\phi\]

Solving the module checking problem for \CTLs specifications can thus
be done by model checking the above \SLref formula, which can be done
in 2\EXPTIME.

%%% Local Variables:
%%% mode: latex
%%% TeX-master: "main"
%%% End:



% \subsection{Synthesis with reactive environments}
% \label{sec-synth-open-env}

% The problem studied in~\cite{kupferman2000open} can be captured by the
% following formula:

% \[\exists\var (S,\var) \forall \varb (E,\varb) \phi\]


%%% Local Variables:
%%% mode: latex
%%% TeX-master: "main"
%%% End:
