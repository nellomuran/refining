Nondeterministic strategies (aka plans or protocols), i.e., strategies that associate to the current history a \emph{set of alternative moves} (instead of one) all of which are ``good'' for the objective of the strategy.
%%

Nondeterministic strategies have been studied in literature in several
context.
%%
Possibly the most relevant area is Discrete Event Control where a
central notion is that of \emph{maximally permissive supervisor}
\cite{WonhamRamadge:SIAMJCO87,Cassandras:BOOK06_DES,Wo14}. This is
supervisor that controls a plant, i.e., allows the plant to do only
certain operations at each point in time. Note that this supervisor
does not says exactly what to do to the plant (as a deterministic
strategy) but in fact tries to leave as much freedom as possible to
plant itself blocking only operations that are unsafe. In fact it is
of interest to be \emph{maximally permissive} wrt the plant. 
Indeed the central result of Discrete Control Theory is that such a
maximally permissive supervisor, i.e., nondeterministic strategy
always exists if the plant and the supervisor specifications are
expressed as regular languages. Such a notion has attracted the interest
of the reasoning about actions community
]\cite{DeGiacomoLM12,BanihashemiGL-AAMAS18} and of the reactive program
synthesis community \cite{pnueli1989synthesisshort,EhlersLTV17}.


Another interesting case is that controllers that orchestrate several components to compose a desired global behaviour \cite{DePS13,DeGiacomoVFAL18}. One way of seeing this is that the controller tries to maintain over time a sort of simulation relation between the desired behavior expressed as a transition system and the Cartesian product of the transitino systems of the components. Also in this case the notion of maximally permissive nondeterministic strategy arises \cite{DePS13}.


Nondeterministic strategies are also of interest in planning \cite{GeffnerBo13}.  Action preconditions themselves can be though of as generating a nondeterministic strategy  as a function that given the state of the domain returns a set of possible actions. 

Also module checking can be considered as dealing with nondeterministic strategies of the environment \cite{kupferman1997module,jamroga2014module} ....

An important problem is that of refining
  such strategies. For instance, given a nondeterministic strategy
  that allows only safe executions, refine it to, additionally,
  eventually reach a desired state of affairs.  In this paper, we show that such
  problems can be solved elegantly in the framework of Strategy Logic
  (SL), a very expressive logic to reason about strategic
  abilities \cite{chatterjee2010strategy,DBLP:journals/tocl/MogaveroMPV14,DBLP:conf/lics/MogaveroMS13,BMMRV17}. Specifically, we introduce a variant of SL with
  nondeterministic strategies and a strategy refinement operator. We
  show that model checking this logic can be done at no additional
  computational cost with respect to standard SL, and can be used to
  solve problems synthesis, synthesis of most permissive strategies,
  module checking, and more.

%%% Local Variables:
%%% mode: latex
%%% TeX-master: "main"
%%% End:
