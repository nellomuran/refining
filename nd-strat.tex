We start with some basic notations, then we recall classic concurrent game structures, nondeterministic
strategies, and the notion of strategy refinement.

\subsection{Notations}
Let $\Sigma$ be an alphabet. A \emph{finite} (resp. \emph{infinite}) \emph{word} over $\Sigma$ is an element
of $\Sigma^{*}$ (resp. $\Sigma^{\omega}$). 
The \emph{length} of a finite word $w=w_{0}w_{1}\ldots
w_{n}$ is $|w|\egdef n+1$, and $\last(w)\egdef w_{n}$ is its last
letter.
Given a finite (resp. infinite) word $w$ and $0 \leq i < |w|$  (resp. $i\in\setn$), we let $w_{i}$ be the
letter at position $i$ in $w$, $w_{\leq i}$ is the prefix of $w$ that
ends at position $i$ and $w_{\geq i}$ is the suffix that starts
at position $i$.
%If $w$ is infinite, we let $w^{i}\egdef w[i,\omega]$.
We write $w\pref w'$ if $w$ is a prefix of $w'$, and $\FPref{w}$ is
the set of finite prefixes of word $w$. 
Finally, 
%$\id$ is the identity relation on the set $X$, 
the domain of a mapping $f$ is written $\dom(f)$.

\subsection{Concurrent game structures}
\label{sec-CGS}

For convenience we fix for the rest of the paper $\APf$, a finite non-empty set of
\emph{atomic propositions}, and $\Agf$, a finite non-empty set of \emph{agents} or
\emph{players}.

\begin{definition}%[\CGS]
  \label{def-CGS}
  A \emph{concurrent game structure} (or
  \CGS) is a tuple
  $\CGS=(\Act,\setpos,\trans,\val,\pos_\init)$ where
   \begin{itemize}
    % \item $\APf$ is a finite non-empty set of \emph{atomic propositions},
    % \item  $\Agf$ is a finite non-empty set of \emph{players},
    \item $\Act$ is a finite non-empty set of \emph{actions},
    \item $\setpos$ is a finite non-empty set of \emph{positions},
   \item $\trans:\setpos\times \Mov^{\Agf}\to \setpos$ is a \emph{transition function}, 
  \item $\val:\setpos\to 2^{\APf}$ is a \emph{labelling function}, and
  \item $\pos_\init \in \setpos$ is an \emph{initial position}.
  \end{itemize}
\end{definition}

% We define the size $|\CGS|$ of a \CGS
% $\CGS=(\Act,\setpos,\trans,\val,\pos_\init,\obsint)$ as the size of
% an explicit encoding of the transition function: $|\CGS|\egdef
% |\setpos|\times |\Act|^{|\Agf|}\times \lceil \log(|\setpos|)\rceil$.
% We may  write $\pos\in\CGS$ for $\pos\in\setpos$.

In a position $\pos\in\setpos$, where atomic propositions $\val(\pos)$
hold, each player $\ag$ chooses an action $\mova\in\Mov$, 
and the game proceeds to position
$\trans(\pos, \jmov)$, where $\jmov\in \Mov^{\Agf}$ stands for the \emph{joint action}
$(\mova)_{\ag\in\Agf}$. Given a joint action
$\jmov=(\mova)_{\ag\in\Agf}$ and $\ag\in\Agf$, we let
$\jmov_{\ag}$ denote $\mova$.
A \emph{finite} (resp. \emph{infinite}) \emph{play} is a finite (resp. infinite)
word $\fplay=\pos_{0}\ldots \pos_{n}$ (resp. $\iplay=\pos_{0} \pos_{1}\ldots$)
such that $\pos_0=\pos_\init$ and for every $i$ such that $0\leq i<|\fplay|-1$ (resp. $i\geq 0$), there exists a joint action $\jmov$
such that $\trans(\pos_{i}, \jmov)=\pos_{i+1}$.
Given two finite plays $\fplay$ and $\fplay'$, we say that $\fplay'$
is a \emph{continuation} of $\fplay$ if $\fplay'\in\fplay\cdot
\setpos^*$, and we write $\setcontinuations{\fplay}$ for the set of
continuations of $\fplay$.

\bam{say that CGS also capture turn-based}

\subsection{Strategy refinement}
\label{sec-ndstrat}
Given a \CGS $\CGS$, a \emph{nondeterministic strategy}, or strategy
for short, for a player is a
function $\strat:\setcontinuations{\pos_\init}\to 2^\Mov\setminus\emptyset$
that maps each finite play in $\CGS$ to a nonempty finite set of
actions that the player may choose from after this finite play.  A strategy $\strat$ is
\emph{deterministic} if for every finite play $\fplay$,
$\strat(\fplay)$ is a singleton.  We let $\setstrat$ denote the set of
all (nondeterministic) strategies, and $\setstratd\subset\setstrat$
the set of deterministic ones (note that these sets depend on the \CGS
under consideration).

Formulas of our logic \SLref will be evaluated at the end of a finite play
$\fplay$ (which can be simply the initial position of the game), and
since \SLref contains only \emph{future-time} temporal operators,
the only relevant part of a strategy $\strat$ when evaluating  a
formula after finite play $\fplay$ is its definition on continuations
of $\fplay$. We thus define the \emph{restriction} of $\strat$ to
$\fplay$ as the restriction of $\strat$ to $\fplay\cdot\setpos^+$, that we
write $\substrat{\strat}:\setcontinuations{\fplay}\to 2^\Mov\setminus\emptyset$.
We will then say that a strategy $\strat$ refines another strategy
$\strat'$ after a finite play $\fplay$ if the first one is more
restrictive than the second one on continuations of $\fplay$. More formally:

\begin{definition}
  Strategy $\strat$ \emph{refines} strategy $\strat'$  after finite
  play $\fplay$, written $\strat \refinesc \strat'$, if for every  $\fplay'\in\setcontinuations{\fplay}$,
  $\substrat{\strat}(\fplay')\subseteq \substrat{\strat'}(\fplay')$.
  We simply say that $\strat$ refines $\strat'$ if it refines it after
  the initial position $\pos_\init$, and in that case we write
  $\strat\refines \strat'$.
\end{definition}

%%% Local Variables:
%%% mode: latex
%%% TeX-master: "main"
%%% End:
