This paper is about nondeterministic strategies (aka plans or protocols), i.e., strategies that associate to the current history a \emph{set of alternative moves} (instead of one) all of which are ``good'' for the objective of the strategy.
%%

Nondeterministic strategies have been studied in literature in several
context.
%%
Possibly the most relevant area is Discrete Event Control
where a central notion is that of \emph{maximally permissive
  supervisor} \cite{}. This is supervisor that controls a plant, i.e.,
allows the plant to do only certain operations at each point in
time. Note that this supervisor does not says exactly what to do to
the plant (as a deterministic strategy) but in fact tries to leave as
much freedom as possible to plant itself blocking only operations that
are unsafe. In fact it is of interest to be \emph{maximally
  permissive} wrt the plant. And indeed the central result of Discrete
Control Theory is that such a maximally permissive supervisor, i.e.,
nondeterministic strategy always exists if the plant and the
supervisor specifications are expressed as regular languages.


Another interesting case is that controllers that orchestrate several components to compose a desired global behaviror \cite{Sardina,Logan}. ONe way of seen this is that the controller tries to maintain overtime a sort of simulation relation between the desired behavior expressed as a triansition system and the cartesan product of the transitino systems of the components.


Nondeterministic strategies are of interest in several context. For example, in planning when the action precondition specification can be seen as a nonseterministi strategywe specify a function that given the state of the domain returns a set of possible actions. Now if we consider the state as summary of the relavant part of the history we can see  


Although not as common as standard deterministic strategies, they are quite common


%%% Local Variables:
%%% mode: latex
%%% TeX-master: "main"
%%% End:
