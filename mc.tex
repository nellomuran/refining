We first recall briefly the syntax and semantics of \QCTLs, to which
we will reduce \SLref.



\begin{definition}%[\QCTLs: Syntax]
  \label{def-syntax-QCTLs}
  The syntax of \QCTLs is defined by the following grammar:
  \begin{align*}
  \phi\egdef &\; p \mid \neg \phi \mid \phi\ou \phi \mid \E \psi \mid
  \existsp[p] \phi\\
    \psi\egdef &\; \phi \mid \neg \psi \mid \psi\ou \psi \mid \X \psi \mid
  \psi \until \psi
\end{align*}
where $p\in\APf$. 
\end{definition}

Again, formulas of type $\phi$ are called \emph{state formulas}, those of type $\psi$
are called \emph{path formulas}, and \QCTLs consists of all the state formulas
defined by the grammar, and   we use standard abbreviation 
$\A\psi \egdef \neg\E\neg\psi$.
% The size $|\phi|$ of a formula $\phi$ is defined inductively as usual, but the
% following case: $|\existsp{\cobs}\phi|\egdef 1 + |\cobs| + |\phi|$.

The models of \QCTLs are classic Kripke structures:

\begin{definition}
A \emph{Kripke structure}, or \KS, over $\APf$ is a tuple 
$\KS=(\setstates,\relation,\lab,\sstate_\init)$ where
\begin{itemize}
\item $\setstates$  is a set of
\emph{states}, 
\item $\relation\subseteq\setstates\times\setstates$ is a
left-total\footnote{\ie, for all $\sstate\in\setstates$, there exists $\sstate'$
such that $(\sstate,\sstate')\in\relation$.} \emph{transition
relation}, 
\item $\lab:\setstates\to 2^{\APf}$ is a \emph{\labeling function} and
\item $\sstate_\init \in \setstates$ is an \emph{initial state}.
\end{itemize}
\end{definition}

A \emph{path} in $\KS$  is an infinite sequence of states
$\spath=\sstate_{0}\sstate_{1}\ldots$ such that
 for all $i\in\setn$,
$(\sstate_{i},\sstate_{i+1})\in \relation$. 
A \emph{finite path} is a finite non-empty prefix of a path.
Similar to continuations of finite plays, given a finite path $\spath$ we write
$\setcontinuations{\spath}$ for the set of finite paths that start
with $\spath$.
We may write $\sstate\in\KS$ for $\sstate\in\setstates$, and we
define the \emph{size} $|\KS|$ of a \KS
$\KS=(\setstates,\relation,\sstate_\init,\lab)$ as its number of states: $|\KS|\egdef
|\setstates|$. 

Since we will interpret \QCTLs on unfoldings of \KS, we now define
infinite trees.

\halfline
\head{Trees}
Let $\Dirtree$ be a finite set of \emph{directions} (typically a set of states). 
An \emph{$\Dirtree$-tree} $\tree$ 
 is a nonempty set of words $\tree\subseteq \Dirtree^+$ such that
 $\bm{(1)}$ there exists $\racine\in\Dirtree$,  called the
    \emph{root} of $\tree$, such that each
    $\noeud\in\tree$ starts with $\racine$ ($\racine\pref\noeud$);
$\bm{(2)}$  if $\noeud\cdot\dir\in\tree$ and $\noeud\cdot\dir\neq\racine$, then
    $\noeud\in\tree$;
 $\bm{(3)}$ if $\noeud\in\tree$ then there exists $\dir\in\Dirtree$ such that $\noeud\cdot\dir\in\tree$.

 The elements of a tree $\tree$ are called \emph{nodes}.  
  If 
 $\noeud\cdot\dir \in \tree$, we say that $\noeud\cdot\dir$ is a \emph{child} of
 $\noeud$.
An $\Dirtree$-tree $\tree$ is \emph{complete} if for every $\noeud \in
\tree$ and  $\dir \in \Dirtree$,  $\noeud \cdot \dir \in \tree$.
A \emph{\tpath} in $\tree$ is an infinite sequence of nodes $\tpath=\noeud_0\noeud_1\ldots$
such that for all $i\in\setn$, $\noeud_{i+1}$ is a child of
$\noeud_i$,
and $\tPaths(\noeud)$ is the set of \tpaths
 that start in node $\noeud$. 

 An \emph{$\APf$-\labeled $\Dirtree$-tree}, or
\emph{$(\APf,\Dirtree)$-tree} for short, is a pair
$\ltree=(\tree,\lab)$, where $\tree$ is an $\Dirtree$-tree called the
\emph{domain} of $\ltree$ and
$\lab:\tree \rightarrow 2^{\APf}$ is a \emph{\labeling}, which maps
each node to the set of propositions that hold there.
 For $p\in\APf$, a \emph{$p$-\labeling} for a  tree is a mapping
$\plab:\tree\to \{0,1\}$ that indicates in which nodes $p$ holds, and
for a \labeled tree $\ltree=(\tree,\lab)$, the $p$-\labeling of $\ltree$ is
the $p$-\labeling $\noeud \mapsto 1$ if $p\in\lab(\noeud)$, 0 otherwise. 
The composition of a \labeled tree $\ltree=(\tree,\lab)$ with a
$p$-\labeling $\plab$ for $\tree$ is defined as
$\ltree\prodlab\plab\egdef(\tree,\lab')$, where
$\lab'(\noeud)=\lab(\noeud)\union \{p\}$ if $\plab(\noeud)=1$, and
$\lab(\noeud)\setminus \{p\}$ otherwise.
A $p$-\labeling for a labelled tree $\ltree=(\tree,\lab)$ is a
$p$-\labeling for its domain $\tree$.
A \emph{pointed labelled tree} is a pair $(\ltree,\noeud)$ where
 $\noeud$ is a node of $\ltree$.

 % \begin{definition}
 %   \label{def-unfolding}
Let $\KS=(\setstates,\relation,\lab,\sstate_\init)$ be a Kripke structure over $\APf$. 
The \emph{tree-unfolding of $\KS$} is the $(\APf,\setstates)$-tree $\unfold{\sstate}\egdef (\tree,\lab')$, where
    $\tree$ is the set
    of all finite  paths that start in $\sstate_\init$, and
    for every $\noeud\in\tree$,
    $\lab'(\noeud)\egdef \lab(\last(\noeud))$.
%  \end{definition}

\begin{definition}%[\QCTLsi semantics]
We define by induction the satisfaction relation $\modelst$ of
\QCTLs. Let   $\ltree=(\tree,\lab)$ be
an $\APf$-\labeled tree, 
$\noeud$  a node and $\tpath$  a path in $\tree$:
% \begingroup
% \addtolength{\jot}{-3pt}
\[
\begin{array}{lcl}
  \ltree,\noeud\modelst p 			& \mbox{ if } & p\in\lab(\noeud)\\[1pt]
  \ltree,\noeud\modelst \neg \phi		& \mbox{ if } & \ltree,\noeud\not\modelst \phi\\[1pt]
  \ltree,\noeud\modelst \phi \ou \phi'		& \mbox{ if } & \ltree,\noeud \modelst \phi \mbox{ or    }\ltree,\noeud\modelst \phi' \\[1pt]
  \ltree,\noeud\modelst \E\psi			& \mbox{ if } & \exists\,\tpath\in\tPaths(\noeud) \mbox{      s.t. }\ltree,\tpath\modelst \psi \\[1pt]
  \ltree,\noeud\modelst \existsp \phi & \mbox{ if }
  & \exists\,\plab \mbox{ a $p$-\labeling for
    $\ltree$ s.t.
    }\\
  & & \quad \quad  \ltree\prodlab\plab,\noeud\modelst\phi\\[1pt]
\ltree,\tpath\modelst \phi 			& \mbox{ if } & \ltree,\tpath_{0}\modelst\phi \\[1pt] 
\ltree,\tpath\modelst \neg \psi 		& \mbox{ if }
&  \ltree,\tpath\not\modelst \psi \\[1pt] 
\ltree,\tpath\modelst \psi \ou \psi'			& \mbox{ if } & \ltree,\tpath \modelst \psi \mbox{ or }\ltree,\tpath\modelst \psi' \\[1pt] 
\ltree,\tpath\modelst \X\psi 				& \mbox{ if } & \ltree,\tpath_{\geq 1}\modelst \psi \\[1pt] 
\ltree,\tpath\modelst \psi\until\psi' 		& \mbox{ if }
& \exists\, i\geq 0 \mbox{ s.t.    }\ltree,\tpath_{\geq
                                                                i}\modelst\psi' \text{ and }\\
  & & \quad \forall j \text{ s.t. }0\leq j <i,\; \ltree,\tpath_{\geq j}\modelst\psi
\end{array}
\]
%\endgroup
\end{definition}

We write $\ltree\modelst\phi$ for $\ltree,\racine\modelst\phi$,
where $\racine$ is the root of $\ltree$.     Given a \KS $\KS$  and a
\QCTLs formula $\phi$, we also write $\KS\modelst\phi$ if
$\unfold[\KS]{\sstate}\modelst\phi$.

\bam{define alternation depth}

\begin{theorem}
  \label{theo-qctls}
  The model-checking problem for \QCTLs is \kEXPTIME[k+1]-complete for
  formulas of alternation depth $k$.
\end{theorem}

\subsection{Reduction to \QCTLs}
\label{sec-reduction}

We use a variant of the reductions presented
in~\cite{DBLP:journals/iandc/LaroussinieM15,DBLP:conf/csl/FijalkowMMR18,BMMRV17,DBLP:conf/kr/MaubertM18,DBLP:conf/ijcai/BouyerKMMMP19},
which transform instances of the model-checking problem for various
strategic logics to (extensions of) \QCTLs.

Let $(\CGS,\Phi)$ be an instance of the \SL
model-checking problem, and assume without loss of generality that
each strategy variable is quantified at most once in $\Phi$. We define an equivalent instance of the
model-checking problem for \QCTLs.

\bam{say that the reduction preserves alternation depth}

Define the \KS $\KS_{\CGS}\egdef(\setstates,\relation,\sstate_{\init},\lab')$ where
\begin{itemize}
\item $\setstates\egdef\{\sstate_{\pos} \mid \pos\in\setpos\}$,
\item $\relation\egdef\{(\sstate_{\pos},\sstate_{\pos'})\mid
  \exists\jmov\in\Mov^{\Agf} \mbox{ s.t. }\trans(\pos,\jmov)=\pos'\}
  \subseteq \setstates^2$,
  \item $\sstate_{\init}\egdef\sstate_{\pos_{\init}}$, and
\item $\lab'(\sstate_{\pos})\egdef\val(\pos)\union \{p_{\pos}\} \subseteq \APf \cup \APv$.
\end{itemize}



For every finite play $\fplay=\pos_{0}\ldots\pos_{k}$, define
the node $\noeud_{\fplay}\egdef \sstate_{\pos_{0}}\ldots \sstate_{\pos_{k}}$ in
$\unfold[\KS_{\CGS}]{\sstate_{\pos_{0}}}$.  Note that the mapping
$\fplay\mapsto\noeud_{\fplay}$ defines a bijection between the set
of finite plays and the set of
nodes in $\unfold[\KS_{\CGS}]{\sstate_{\init}}$. % It also puts in
% bijection $\setcontinuations{\fplay}$ and $\{\noeud_\spath\mid \spath\in\setcontinuations{\}\}$


\halfline
\head{Constructing the \QCTLs formulas $\tr[f]{\phi}$}
 We now describe how to transform an \SLref formula $\phi$ and a partial
function $f:\Agf \partialto  \Varf$ into a \QCTLs
formula $\tr[f]{\phi}$ (that will also depend on $\CGS$).
Suppose that $\Mov=\{\mov_{1},\ldots,\mov_{\maxmov}\}$, and define
$\tr[f]{\phi}$ and $\trp[f]{\psi}$ by mutual induction on state and path formulas. 
The base cases are as follows:
$\tr[f]{p} 		 \egdef p$ and $\trp[f]{\phi} \egdef
\tr[f]{\phi}$. Boolean and temporal operators are simply obtained by
distributing the translation:
$\tr[f]{\neg \phi} 	 \egdef \neg \tr[f]{\phi}$, $\trp[f]{\neg
  \psi} \egdef \neg \trp[f]{\psi}$,
$\tr[f]{\phi_1\ou\phi_2}  \egdef \tr[f]{\phi_1}\ou\tr[f]{\phi_2}$,
$\trp[f]{\psi_1\ou\psi_2}  \egdef \trp[f]{\psi_1}\ou\trp[f]{\psi_2}$,
$\trp[f]{\X\psi}  \egdef \X\trp[f]{\psi}$ and $\trp[f]{\psi_1\until\psi_2}  \egdef \trp[f]{\psi_1}\until\trp[f]{\psi_2}$.


% \begin{align*}
% \tr[f]{p} 		& \egdef p  \\
% \tr[f]{\neg \phi} 	& \egdef \neg \tr[f]{\phi}\\
% \tr[f]{\phi_1\ou\phi_2} & \egdef \tr[f]{\phi_1}\ou\tr[f]{\phi_2}. \\
% \end{align*}
%where $f_i$ is defined to be $f$ restricted to domain $\freeFun{\phi_i} \cap \Agf$.
We continue with the strategy quantifiers:
\[
  \begin{array}{lrl}
& \tr[f]{\Estratnd\phi}	& \egdef  \exists
                                  p_{\mov_{1}}^{\var}\ldots
                                  \exists^{\trobs{\obs}}
                                  p_{\mov_{\maxmov}}^{\var}. \phistrat
                                  \et \tr[f]{\phi}\\[5pt]
  \mbox{where} &  \phistrat & \egdef \A\always
                              \bigou_{\mov\in\Mov}p_{\mov}^{\var},
                              \quad\mbox{ and}\\[10pt]
& \tr[f]{\Estratd\phi}	& \egdef  \exists
                                  p_{\mov_{1}}^{\var}\ldots
                                  \exists^{\trobs{\obs}}
                                  p_{\mov_{\maxmov}}^{\var}. \phistratdet
                                  \et \tr[f]{\phi}\\[5pt]
  \mbox{where} & \phistratdet & \egdef
\A\always\bigou_{\mov\in\Mov}(p_{\mov}^{\var}\et\biget_{\mov'\neq\mov}\neg
p_{\mov'}^{\var}).                              
\end{array}
\]

The intuition is that for each possible action $\mov\in\Mov$, an
existential quantification on the atomic proposition $p_{\mov}^{\var}$
``chooses'' for each  node $\noeud_{\fplay}$ of the tree 
$\unfold[\KS_{\CGS}]{\sstate_{{\pos_0}}}$ whether strategy $\var$
allows action $\mov$ in $\fplay$ or not. 
$\phistrat$  checks that at least one action is allowed in each
node, and thus that atomic propositions
$p_{\mov}^{\var}$ indeed define a (nondeterministic) strategy.
$\phistratdet$ instead ensures that \emph{exactly one} action is chosen for strategy $\var$  in each finite
play, and thus that atomic propositions
$p_{\mov}^{\var}$  characterise a
deterministic strategy.

For strategy refinement, the translation is as follows:
\[\tr[f]{\var \refines \varb} \egdef \A\always\biget_{\mov\in\Mov} p_\mov^\var
  \impl p_\mov^{\varb}.\]

Here are the remaining cases:
\[
\begin{array}{lrl}
& \tr[f]{\bind{\var}\phi}	& \egdef \tr[{f[\ag\mapsto \var]}]{\phi} \quad\quad
                          \text{for }\var\in\Varf\union\{\unb\}  \\[5pt]
\mbox{and} & \tr[f]{\Eout\psi}	& \egdef \E\,(\psiout[f] \wedge
                                 \trp[f]{\psi}), \mbox{ where}
\end{array}
\]

\begin{align*}
\psiout[f]  \egdef \always
  \bigou_{\pos\in\setpos}\Big( & p_{\pos}\, \et  \\
  & \bigou_{\jmov\in\Mov^{\Agf}} 
  (\biget_{\ag\in\dom(f)}p_{\jmov_{\ag}}^{f(\ag)}\et \X\,
                          p_{\trans(\pos,\jmov)}) \Big ).  
\end{align*}

                      
$\psiout[f]$ checks that  each player $\ag$ in the domain of $f$
follows the strategy coded by the $p_\act^{f(\ag)}$. 

   
To prove the correctness of the translation we need some additional
definitions. First, given a strategy $\strat$ and
a strategy variable $\var$ we
let  $\stratlab{\var}\egdef\{\plab[{p_\act^\var}]\mid
\act\in\Act\}$ be the family of $p_\act^\var$-\labelings for tree
$\unfold[\KS_{\CGS}]{}$ defined as follows: for each
finite play $\fplay$ and $\act\in\Act$,
we let $\plab[{p_\act^\var}](\noeud_\fplay)\egdef1$ if $\act\in\strat(\fplay)$, 0 otherwise.
For a \labeled tree $\ltree$ with same domain as
$\unfold[\KS_{\CGS}]{}$ we write $\ltree\prodlab \stratlab{\var}$ for
$\ltree\prodlab \plab[{p_{\act_1}^\var}]\prodlab\ldots\prodlab \plab[{p_{\act_\maxmov}^\var}]$.

Second, given an infinite play $\iplay$ and a point $i\in\setn$, we let
$\tpath_{\iplay,i}$ be the infinite path in
$\unfold[\KS_{\CGS}]{\sstate_{\pos_\init}}$ that starts in node
$\noeud_{\iplay_{\leq i}}$ and is defined as
$\tpath_{\iplay,i}\egdef\noeud_{\iplay_{\leq i}}\noeud_{\iplay_{\leq
    i+1}}\noeud_{\iplay_{\leq i+2}}\ldots$

Finally, we say that a partial   function $f:\Agf\partialto\Varf$ is
\emph{compatible} with an assignment $\assign$ if
 $\dom(\assign)\inter \Agf=\dom(f)$ and  for all $a \in \dom(f)$,  $\assign(a) = \assign(f(a))$.

 \begin{proposition}
   \label{prop-redux}
   For every  state subformula $\phi$ and path subformula $\psi$ of
   $\Phi$, finite play $\fplay$, infinite play $\iplay$, point
   $i\in\setn$, for every  assignment $\assign$ variable-complete for
   $\phi$ (resp. $\psi$) and
partial   function $f:\Agf\partialto\Varf$ compatible with $\assign$, assuming
   also that no $\var_i$ in $\dom(\assign)\inter \Varf=\{\var_1,\ldots,\var_k\}$ is
 quantified in $\phi$ or $\psi$, we have
   \begin{align*}
\CGS,\assign,{\fplay}\models\phi && \mbox{iff} &&
  \unfold[\KS_{\CGS}]{\sstate_{\pos_\init}}\prodlab
  \stratlab[\assign(\var_1)]{\var_1}\prodlab\ldots \prodlab
  \stratlab[\assign(\var_k)]{\var_k},\noeud_{\fplay} \modelst
                                                              \tr[f]{\phi}\\
\CGS,\assign,{\iplay},i\models\psi && \mbox{iff} &&
  \unfold[\KS_{\CGS}]{\sstate_{\pos_\init}}\prodlab
  \stratlab[\assign(\var_1)]{\var_1}\prodlab\ldots \prodlab
  \stratlab[\assign(\var_k)]{\var_k},\tpath_{\iplay,i} \modelst
  \trp[f]{\psi}     
   \end{align*}

  In addition, $\KS_{\CGS}$ is of size linear in
$|\CGS|$, and $\tr[f]{\phi}$ and $\trp[f]{\psi}$ are of size linear in $|\CGS|^2+|\phi|$.
 \end{proposition}

 \begin{proof}
   The proof is by induction on $\phi$.
We detail the case for binding,  strategy quantification, strategy refinement and
outcome quantification, the others follow simply by definition of
$\KS_{\CGS}$ for atomic propositions and induction hypothesis for
remaining cases.

\halfline
For $\phi=\var\refines\varb$,
assume that $\CGS,\assign,{\fplay}\models\var\refines\varb$.
First, observe that since $\assign$ is variable-complete for $\phi$,
$\var$ and $\varb$ are in $\dom(\assign)$.
Now we have that 
$\substrat{\assign(\var)}(\fplay')\subseteq\substrat{\assign(\varb)}(\fplay')$
for every $\fplay'\in \setcontinuations{\fplay}$. By definition of
$\stratlab[\assign(\var)]{\var}=\{\plab[{p_\act^\var}]\mid
\act\in\Act\}$ and
$\stratlab[\assign(\varb)]{\varb}=\{\plab[{p_\act^\varb}]\mid
\act\in\Act\}$, it follows that
 for each $\act\in\Act$ and  $\fplay'\in\setcontinuations{\fplay}$, if
 $\plab[{p_\act^\var}](\fplay')=1$, then
 $\plab[{p_\act^\varb}](\fplay')=1$, and thus

\[\unfold[\KS_{\CGS}]{\sstate_{\pos_\init}}\prodlab
  \stratlab[\assign(\var)]{\var}\prodlab
  \stratlab[\assign(\varb)]{\varb}\models\A\always\biget_{\mov\in\Mov} p_\mov^\var
  \impl p_\mov^{\varb}\]
Because the labellings $\stratlab[\assign(\var)]{\var}$ touch distinct
sets of atomic propositions for each variable $\var$ in
$\dom(\assign)\cap \Varf$, we can conclude this direction.

For the other direction let $\ltree=\unfold[\KS_{\CGS}]{\sstate_{\pos_\init}}\prodlab
\stratlab[\assign(\var_1)]{\var_1}\prodlab\ldots \prodlab
\stratlab[\assign(\var_k)]{\var_k}$ and assume that 
\[\ltree,\noeud_\fplay\models \A\always\biget_{\mov\in\Mov} p_\mov^\var
  \impl p_\mov^{\varb}.\]
This implies that for every $\fplay'\in\setcontinuations{\fplay}$, \[\ltree,\noeud_{\fplay'}\models\biget_{\mov\in\Mov} p_\mov^\var
  \impl p_\mov^{\varb},\] and thus $\substrat{\assign(\var)}$ refines $\substrat{\assign(\varb)}$.

\halfline
For $\phi=\bind{\var}\phi'$, we have
$\CGS,\assign,{\fplay}\models\bind{\var}\phi'$ if and only if 
$\CGS,\assign[\ag\mapsto\assign(\var)],{\fplay}\models\phi'$.
The result follows by using the induction hypothesis with assignment
$\assign[\ag\mapsto\var]$ and function
 $f[a\mapsto\var]$. This is possible because $f[a\mapsto\var]$ is compatible with $\assign[\ag\mapsto\var]$: indeed
 $\dom(\assign[\ag\mapsto\var])\inter\Agf$ is equal to
 $\dom(\assign)\inter\Agf \union \{a\}$ which, by assumption, is equal
 to $\dom(f) \union \{a\}=\dom(f[a\mapsto
 \var])$. Also
 by assumption, for all $a'\in\dom(f)$, $\assign(a')=\assign(f(a'))$, and 
by definition $\assign[a\mapsto \assign(\var)](a)=\assign(\var)=\assign(f[a\mapsto\var](a))$.


\halfline
For $\phi=\Estratnd\phi'$, assume first that
$\CGS,\assign,{\fplay}\models\Estratnd\phi'$. There exists a
nondeterministic strategy $\strat$ such that
 \[\CGS,\assign[\var\mapsto \strat],\fplay\models \phi'.\] Since $f$ is
 compatible with $\assign$, it is also compatible with assignment
 $\assign'=\assign[\var\mapsto \strat]$. By assumption, no variable in
 $\{\var_1,\ldots,\var_k\}$ is quantified in $\phi$, so that $\var\neq
 \var_i$ for all $i$, and thus $\assign'(\var_i)=\assign(\var_i)$ for
 all $i$; and because no strategy variable is
 quantified twice in a same formula,
 $\var$ is not quantified in $\phi'$, so that no variable in
 $\{\var_1,\ldots,\var_k,\var\}$ is quantified in $\phi'$.
 By induction hypothesis 
   \[\unfold[\KS_{\CGS}]{\sstate_{\pos_\init}}\prodlab
   \stratlab[\assign'(\var_1)]{\var_1}\prodlab\ldots \prodlab
   \stratlab[\assign'(\var_k)]{\var_k}\prodlab   \stratlab[\assign'(\var)]{\var},\noeud_{\fplay}
   \modelst \tr[f]{\phi'}.\]

It follows that   \[\unfold[\KS_{\CGS}]{\sstate_{\pos_\init}}\prodlab
  \stratlab[\assign'(\var_1)]{\var_1}\prodlab\ldots \prodlab
   \stratlab[\assign'(\var_k)]{\var_k},\noeud_{\fplay}
   \modelst \exists^{\trobs{\obs}} p_{\mov_{1}}^{\var}\ldots
   \exists   p_{\mov_{\maxmov}}^{\var}. \phistrat\et\tr[f]{\phi'}.\]
 Finally, since $\assign'(\var_i)=\assign(\var_i)$ for all $i$, we
 conclude that 
 \[\unfold[\KS_{\CGS}]{\sstate_{\pos_\init}}\prodlab
   \stratlab[\assign(\var_1)]{\var_1}\prodlab\ldots \prodlab
   \stratlab[\assign(\var_k)]{\var_k},\noeud_{\fplay}
   \modelst \tr[f]{\Estratnd\phi'}.\]

For the other direction, assume
that
\[\unfold[\KS_{\CGS}]{\sstate_{\pos_\init}}\prodlab
  \stratlab[\assign(\var_1)]{\var_1}\prodlab\ldots \prodlab
  \stratlab[\assign(\var_k)]{\var_k},\noeud_{\fplay} \modelst
  \tr[f]{\phi},\] and recall that
$\tr[f]{\phi}=\exists^{\trobs{\obs}} p_{\mov_{1}}^{\var}\ldots
\exists^{\trobs{\obs}}
p_{\mov_{\maxmov}}^{\var}. \phistrat\et\tr[f]{\phi'}$.  Write
$\ltree=\unfold[\KS_{\CGS}]{\sstate_{\pos_\init}}\prodlab
\stratlab[\assign(\var_1)]{\var_1}\prodlab\ldots \prodlab
\stratlab[\assign(\var_k)]{\var_k}$. There exist
 $\plab[{p_\act^\var}]$-\labelings such that
\[\ltree\prodlab  \plab[{p_{\act_1}^\var}]\prodlab\ldots\prodlab \plab[{p_{\act_\maxmov}^\var}]
  \modelst \phistrat\et\tr[f]{\phi'}.\]
By $\phistrat$, these \labelings  
 code for a strategy $\strat$. Let
 $\assign'=\assign[\var\mapsto \strat]$. For all $1\leq i\leq k$, by
 assumption $\var\neq \var_i$, and thus $\assign'(\var_i)=\assign(\var_i)$.
 The above can thus be rewritten
 \[\unfold[\KS_{\CGS}]{\sstate_{\pos_\init}}\prodlab
\stratlab[\assign'(\var_1)]{\var_1}\prodlab\ldots \prodlab
\stratlab[\assign'(\var_k)]{\var_k}\prodlab  \stratlab[\assign'(\var)]{\var}
  \modelst \phistrat\et\tr[f]{\phi'}.\]
 By induction hypothesis we have
$\CGS,\assign[\var\mapsto
\strat],\fplay\models \phi'$, hence $\CGS,\assign,\fplay\models
\Estratnd\phi'$.

\halfline
For $\phi=\Estratd\psi$, the proof is similar, using $\phistratdet$
instead of $\phistrat$.

 
\halfline
For $\phi=\Eout\psi$,
assume first that $\CGS,\assign,{\fplay}\models\E\psi$. 
There exists a play $\iplay\in\out(\assign,\fplay)$ s.t.
$\CGS,\assign,\iplay,|\fplay|-1\modelsSL \psi$. By induction
hypothesis,
\[\unfold[\KS_{\CGS}]{\sstate_{\pos_\init}}\prodlab
  \stratlab[\assign(\var_1)]{\var_1}\prodlab\ldots \prodlab
  \stratlab[\assign(\var_k)]{\var_k},\tpath_{\iplay,|\fplay|-1} \modelst
  \trp[f]{\psi}.\] Since $\iplay$ is an outcome of $\assign$, each agent $a\in\dom(\assign)\inter\Agf$ 
follows strategy $\assign(a)$ in $\iplay$.
Because  $\dom(\assign)\inter \Agf=\dom(f)$ and for all $a \in \dom(f)$,
  $\assign(a) = \assign(f(a))$, each agent $a\in\dom(f)$ follows
the  strategy $\assign(f(a))$, which is coded by atoms
$p_\mov^{f(\ag)}$ in the translation of $\Phi$. Therefore $\tpath_{\iplay,|\fplay|-1}$ also
satisfies $\psiout$, hence \[\unfold[\KS_{\CGS}]{\sstate_{\pos_\init}}\prodlab
  \stratlab[\assign(\var_1)]{\var_1}\prodlab\ldots \prodlab
  \stratlab[\assign(\var_k)]{\var_k},\tpath_{\iplay,|\fplay|-1} \modelst
  \psiout \et   \trp[f]{\psi},\] and we are done.

  For the other direction, assume that 
  \[\unfold[\KS_{\CGS}]{\sstate_{\pos_\init}}\prodlab
  \stratlab[\assign(\var_1)]{\var_1}\prodlab\ldots \prodlab
  \stratlab[\assign(\var_k)]{\var_k},\noeud_\fplay \modelst
  \E(\psiout[f] \et   \trp[f]{\psi}).\]
There exists a path $\tpath$ in $\unfold[\KS_{\CGS}]{\sstate_{\pos_\init}}\prodlab
  \stratlab[\assign(\var_1)]{\var_1}\prodlab\ldots \prodlab
  \stratlab[\assign(\var_k)]{\var_k}$ starting in
node $\noeud_\fplay$ that satisfies both $\psiout[f]$ and $\trp[f]{\psi}$.
By construction of $\KS_{\CGS}$ there exists an infinite play $\iplay$
such that $\iplay_{\leq |\fplay|-1}=\fplay$ and $\tpath=\tpath_{\iplay,|\fplay|-1}$.
By induction hypothesis, $\CGS,\assign,\iplay,|\fplay|-1 \modelsSL \psi$.
Because $\tpath_{\iplay,|\fplay|-1}$ satisfies $\psiout[f]$, $\dom(\assign)\inter \Agf=\dom(f)$, and for all $a \in \dom(f)$,
  $\assign(a) = \assign(f(a))$, it is also the case that
  $\iplay\in\out(\assign,\fplay)$, 
hence  $\CGS,\assign,\fplay \modelsSL \Eout\psi$.

% The size of $\KS_{\CGS}$, $\tr[f]{\phi}$ and $\trp[f]{\psi}$ are easily verified.

%$\tr[f]{\Estrato{\obs}\phi}$ is of size
% $O(|\CGS|^k+|\tr[f]{\phi}|)$, with $k=1$ for nondeterministic
% strategies and $k=2$ for deterministic ones.
% and it is of size $O(|\setpos|\times |\Act|^{|\Agf|}\times
% |\Agf|)\leq O(|\CGS|\times |\Agf|) \leq O(|\CGS|^2)$.
% $\tr[f]{\Eout\psi}$ is thus of size $O(|\CGS|^2+\trp[f]{\psi})$.
 \end{proof}
 
 Applying Proposition~\ref{prop-redux} to the sentence $\Phi$, $\fplay=\pos_\init$, any assignment $\assign$, and
 the empty function $\emptyset$, we get:
 \[\CGS \models \Phi \quad \mbox{if and only if}\quad
\unfold[\KS_{\CGS}]{} \models
 \tr[\emptyset]{\Phi}.\]


  
%%% Local Variables:
%%% mode: latex
%%% TeX-master: "main"
%%% End:
