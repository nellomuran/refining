see:
\begin{itemize}
\item Julien Bernet, David Janin, and Igor Walukiewicz. Permissive
  strategies:from parity games to safety games

  $\rightarrow$ shows that in parity games we can compute winning nondeterministic
  strategies that are more permissive than all memoryless winning strategies
\item Patricia Bouyer, Marie Duflot, Nicolas Markey, and Gabriel
  Renault. Measuring permissivity in finite games

  $\rightarrow$ quantitative notion of permissiveness: to each action in
  each state is associated a number representing the \emph{penalty}
  incurred by a strategy that disallows this action. Permissiveness of
  a strategy is then the supremum over all outcomes of the mean
  penalty along this outcome.
  
\item Patricia Bouyer, Nicolas Markey, Jörg Olschewski and Michael
  Ummels. Measuring permissiveness inparity games: Mean-payoff parity
  games revisited

  $\rightarrow$ generalisation to parity games
  
\item Sophie Pinchinat and Stéphane Riedweg. You can always compute
  maximally permissive controllers under partial observation when they
  exist
  
$\rightarrow$   for branching mu-calculus

  
\item Michael Luttenberger. Strategy iteration using non-deterministic
  strategies for solving parity games  

$\rightarrow$ solving parity games with nondeterministic strategies

\end{itemize}


\subsection{Maximal permissive synthesis}



\subsection{Plan B}


%%% Local Variables:
%%% mode: latex
%%% TeX-master: "main"
%%% End:
