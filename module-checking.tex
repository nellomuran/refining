Module checking~\cite{kupferman2001module} is a generalisation of model checking to the setting
of open systems, \ie, systems that interact with an environment. The
idea is to check that the system satisfies a given property, specified
for instance in \LTL or \CTLs, for any possible environment in which
it is used.

In this setting, the environment is seen as an entity that can cut
some transitions of the system. One particular environment can thus be
modelled as a nondeterministic strategy that defines which transitions
are allowed, and the module checking problem for a property $\phi$ can
be written as follows in \SLref, where $E$ is the environment:

\[\Astratnd (E,\var) \,u\phi\]

Solving the module checking problem for \CTLs specifications can thus
be done by model checking the above \SLref formula, which can be done
in 2\EXPTIME.

%%% Local Variables:
%%% mode: latex
%%% TeX-master: "main"
%%% End:
